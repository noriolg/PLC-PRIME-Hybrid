\section{Objectives}

This project has three main objectives. The PRIME Hybrid simulation tool will be developed to reach meaningful conclusions in all three goals.

\begin{itemize}
  \item Consider the viability of a PRIME Hybrid solution (PLC + RF).
  \item Understand the possible discrepancies and gray areas that arise when combining PLC and RF standards. A potential approach to these unclear points will be proposed.
  \item Examine the effectiveness of a PRIME Hybrid solution by simulating a network under different working conditions.
\end{itemize}

The development of the simulation tool is not an objective in itself. It will be crucial in reaching the objectives listed above but it will be considered as a means to an end, a tool used to reach the main conclusions. Further explanation is provided below regarding how to achieve the proposed goals.


\begin{description}
    \item \textbf{Viability of Hybrid solution.} Before diving fully into the development of the simulation tool, an analysis will be done regarding the compatibility of the different standards. The possibility of combining the protocol stacks of PLC and RF needs to be studied and understood. To achieve this, an explanation is provided regarding the newest PRIME standard (v1.4) and how to include an RF PHY layer in its protocol stack.
    \item \textbf{Gray areas in standards.} The merging of PLC and RF is expected to not be seamless. Any decisions taken in the process that are not fully detailed in current standards will be laid out and discussed. For example, if the node's choice to use PLC or RF is dynamic, several possible decision parameters will be considered.
    \item \textbf{Effectiveness of Hybrid solution.} Simulations will help measure the results of hybrid communications and how they compare to PLC and RF individually. Some expected results are network latencies, throughput and availability. 
\end{description}

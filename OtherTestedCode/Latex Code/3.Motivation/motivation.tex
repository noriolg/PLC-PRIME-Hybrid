%% OK con Grammarly

\section{Motivation}

\subsection{Metering evolution in the Spanish market}
In December 2007, the Spanish government approved the required legislation \cite{boe_renovacion_contadores_2007} "by which all electrical invoices from January 2008 onward must be modernized". In this document, the Government imposes on utility companies the duty to replace all qualifying electricity meters. The measuring equipment would qualify for replacement if contracted power was lower than 15kW\footnote{Average contracted power for a Spanish household is 4kW with an average monthly consumption of 270kWh \cite{ree_consumo_medio}}.

The reason behind this metering overhaul was to install more up-to-date electricity meters. These meters would have hourly discrimination and remote management capabilities. The legislation was updated again in 2012 \cite{boe_renovacion_contadores_2012} to revisit the intermediate milestones while maintaining the final deadline. The final schedule and milestones were: substituting 35\% of meters before 2015; another 35\% before 2017; and the final 30\% before 2019.

According to a report published by \textit{Comisión Nacional de los Mercados y la Competencia (CNMC)}\footnote{Spanish equivalent of European Securities and Markets Authority (ESMA)} the outcome of this metering renovation initiative was a success. By December 31st 2018, 99.04\% of new meters had already been installed \cite{cnmc_renovacion_contadores}. This modernization brought several new crucial tools to the table. The most prominent ones are the automatic readings of users' electric consumption, and more flexible tariff offers (for example with hourly price discrimination).

Nowadays, energy companies are faced with new challenges to improve the overall performance of the electrical grid and increase their operating efficiency. Players must keep updating and renewing their equipment to achieve higher network efficiency and operating flexibility. The final goal is to have a grid with bidirectional communication, automated operations control, and the ability to efficiently collect information generated by consumers and distributed generation units connected to the network. To face these upcoming challenges, the Spanish Government has kept-up legislative pressure to innovate. Electric meters must be renewed again before February 2025 \cite{boe_renovacion_contadores_2020}.

\subsection{Scope of the study within the metering evolution}
At this point is where the present study comes into play. Energy utilities are currently considering different network improvement options. One of these alternatives is the introduction of PRIME Hybrid technology as a means of communicating with end-user meters. In addition to studying the case of the Spanish grid - where PLC communications have already been successfully introduced - PRIME Hybrid will also be tested for other geographies.

Network topologies for Brazil and the United States will be simulated. These countries have fundamentally different characteristics to those of the Spanish electrical system.  For example, where PLC installation is concerned, Spain was an attractive network for this kind of rollout. Spain's fit with PLC technology is mainly due to the high client to substation ratio (around 100 clients per substation)\footnote{A higher number of supply points per substation significantly reduces the scale of the required rollout and its total cost. In this context, substation is referring to MV/LV transformers which are sometimes called \textit{secondary} substations}. Higher ratios imply less capital expenditures per household when deploying the technology which decreases the overall cost of updating the whole grid. This ratio is lower in both the US, and Brazil (the former with around 2 clients per substation and the latter sitting in between the US and Spain) \cite{conversacion_nico_arcauz}. In these cases, RF may provide a  more viable alternative to PLC due to potentially easier installations, higher ranges and lower costs. A combination of the two technologies (PRIME Hybrid) may be the most suitable choice. This is the starting point for our analysis and the motivation to create a simulation tool with capabilities to obtain results comparing RF, PLC and the combination of the two in PRIME Hybrid. These comparisons will be established across different network scenarios and topologies.
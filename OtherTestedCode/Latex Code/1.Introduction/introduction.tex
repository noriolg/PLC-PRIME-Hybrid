\section{Introduction}
\subsection{The electrical grid's current trajectory}
The electrical grid is a system composed of many interconnected pieces. Each of these pieces interacts with the others and contributes to achieving the network's ultimate goal: managing energy supply and demand to accommodate individual demand schedules with centralized and distributed generation. This system is highly complex, and its three main stages - generation, transmission, and distribution - are continuously evolving.

The electrical system has a global presence. However, it cannot be envisioned as one homogeneous unit, rather as a set of interconnected networks. Even though they have different characteristics (topologies, physical magnitudes, standards) there are some challenges and threats that are common to all networks. 


As described in \cite{the_new_frontier_of_smart_grids}, these challenges include planning inefficiencies, periodic consumption patterns with high peak-valley differences, and congestion and loss in transmission and distribution. Most of these boil down to the one factor that distinguishes electrical energy from other commodities: electricity cannot be stored efficiently in large quantities, thus supply must equal demand at all times. From this fact, several problems arise which require that the whole system be resilient and managed with low latencies. Resilience is key because any failure in the system has immediate consequences that can be felt at other points in the network. Low latency allows decisions to be made with up-to-date information, facilitates fault correction, and improves automation capabilities \cite{telco_ntwks_for_the_smart_grid}.

These problems are not new to the industry. However, the scale at which they are now occurring and other factors, such as the increasing share of renewable energy generation,  require the incorporation of new solutions. The set of new solutions and improvements designed to improve the network - which are mainly based on telecommunication services - is nowadays known as the \textit{Smart Grid}. 


\subsection{Low-voltage distribution networks}\label{section_introduction_subsection_lv_networks}
Special attention will be dedicated to de distribution stage of the electrical network. Particularly, low-voltage (LV) distribution networks. This part of the network consists of lines with voltage levels below 1kV. They are the most diverse part of the network as its topology highly depends on the individual population and service area \cite{telco_ntwks_for_the_smart_grid}. For now, it is important to state that MV/LV transformers\footnote{Device that reduces voltage to adequate levels for distribution amongst households, small businesses and medium-sized industrial customers} are part of the distribution networks and are feeding LV lines, connecting them to the MV grid. Thus, given the network's topology, secondary substations are placed in a prime location to boost MV and LV digitization. 

Key Smart Grid applications at LV distribution include Demand Response (DR) and Automatic Meter Management (AMM). Both of these features make use of bidirectional communications between utilities and consumers. DR uses these interactions to facilitate more efficient generation and consumption patterns. AMR allows utilities to collect information automatically from consumers by communicating with electricity meters.

This communication can be done with different technologies such as powerline communication (PLC) or wireless radio-frequency communications (RF). PLC is currently one of the most widespread communication technologies for LV networks. PLC's biggest advantage is that information travels through an already established medium. This channel is pre-existing powerlines, which means that no additional infrastructure rollout is required \cite{primeWhitePaper}.  However, it is not the best technology for all scenarios due to the possibility of noisy channels, non-dense populations and non-existing or unsuitable electric rollouts. In such scenarios, RF communication may be the superior technology. A few reasons why RF may be more suitable in some instances are emission regulations, and high variability of attenuation depending on the communication scenario. High regulations ensure that communications take place in a controlled noise environment whereas a high range of possible attenuation means that RF will be suitable in at least some scenarios.


\subsection{Focus of the study}\label{section_introduction_subsection_focus_of_the_study}
It is at this point where this study will focus. Both technologies have provided positive results independently \cite{prime_hybrid_solution_combines_plc_and_rf}. However, they both have their weaknesses. The merging of both of them in a hybrid solution provides a more flexible and reliable network. PRIME Alliance\footnote{PoweRline Intelligent Metering Evolution} has merged them into a hybrid standard solution: PRIME Hybrid.

We will delve deeper into this standard set by PRIME Alliance in section \ref{section_state_of_the_art}. For now, it can be established that this paper aims to study the viability of the PRIME Hybrid solution for different LV network topologies. To achieve this, a simulation framework based on the PRIME simulator described in \cite{tesis_jmatanza} will be developed. The simulator will be expanded to include the newest version of PRIME specifications and RF capabilities. Finally, an economic comparison will be done with other rollout alternatives (such as the Wi-SUN alliance and/or LTE).

The expected results are a comprehensive comparison of PRIME Hybrid performance across different scenarios and a high-level analysis of the cost-benefit structure of the different rollout alternatives.